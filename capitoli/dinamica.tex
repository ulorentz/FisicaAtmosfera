\chapter{Dinamica dell'Atmosfera}
L'equazione che regola la relazione tra forze e pressione, e quindi la dinamica dell'atmosfera, è
\begin{equation}\label{fpress}
\frac{\vec{F}_p}{m}=-\frac{1}{\rho}\nabla P.
\end{equation}

Nel capitolo precedente è stato analizzato l'andamento della componente verticale del gradiente di pressione, il cui scopo è mantenere in equilibrio l'atmosfera; è lecito chiedersi, invece, quale sia la causa delle componenti orizzontali.

Generalmente, diverse temperature causano variazioni di pressione, ovvero
\begin{equation*}
\nabla_{orizz} T \Rightarrow \nabla_{orizz} P
\end{equation*}
questo fenomeno è detto \emph{baroclinicità}. Questo si configura in gradienti di pressione originati da gradienti di temperatura. Gli ultimi sono dovuti, ad esempio, a diverse composizioni del terreno dotati di capacità termiche differenti; a diverse esposizioni solari; a bacini idrici, ecc\ldots  È da notare che su grandi scale si ha generalmente $\frac{\partial T}{\partial y} > \frac{\partial T}{\partial x}$.\\

Per comprendere l'andamento del gradiente di pressione orizzontale su grandi scale, può essere utile studiare le carte delle isobare. Da queste, infatti, si osservano alcune strutture tipiche, ovvero i cicloni, gli anticicloni, le saccature e i cunei. (INSERIRE FIGURE).

Il ciclone è una struttura circolare che si forma attorno ad un centro di bassa pressione: dalle carte si osserva che i venti seguono le isobare circolando in senso antiorario, con una leggera tendenza alla convergenza verso il centro. 

L'anticiclone si sviluppa in zone di alta pressione, i venti circolano in senso orario, seguendo le isobare, con una leggera tendenza a divergere dal centro.
\section{Forze non inerziali: origine delle strutture circolari}
Si ipotizzi che vi sia solo il gradiente di pressione, in tal caso sul piano orizzontale vi è una forza perpendicolare alle isobare e diretta verso la bassa pressione la cui formula è
\begin{equation}
\frac{(\vec{F}_P)_{horiz}}{m}=-\frac{1}{\rho}\nabla P.
\end{equation}

A scopo di esempio, si supponga la presenza di un $\delta P = 8$ hPa per un $\delta x=400$km, allora si ha: $\frac{\partial P}{\partial x}=2\cdot10^{-3}$Pa/m. Inserendo nella formula precedente, si ottiene in modulo $\left(\frac{\vec{F}_P}{m}\right)_x = -\frac{1}{\rho}\frac{\partial P}{\partial x}\sim 2\cdot 10^{-3}$m/s$^2$. Se non vi fossero altre forze, quest'ultima non sarebbe contrastata e, spostando masse d'aria, andrebbe ad appianare i gradienti di pressione causati da gradienti di temperatura, uniformando l'atmosfera. 

Nel precedente ragionamento si è trascurato il fatto che la Terra non è un sistema di riferimento inerziale e, pertanto, sono presenti ulteriori forze.

(INSERIRE FIGURA TERRA E SISTEMA RIFERIMENTO)
Un punto nel sistema di riferimento terrestre (non inerziale) detto $O'$, rispetto ad un sistema di riferimento inerziale è:
\begin{align}
	\vec{R}&=\vec{OP}(t)=\vec{O'P}(t)+\vec{OO'}\\
	\vec{a}_{assoluta}&=\vec{a}_{rel}+2\vec{\Omega}\times\vec{v}_r+\vec{\Omega}\times(\vec{\Omega}\times\vec{R})
\end{align}
dove $\vec{a}_{rel}$ è quella del sistema non inerziale. Segue che
\begin{align}
	\frac{\sum \vec{F}}{m}=\vec{a}_{as}=\vec{a}_{rel}+2\vec{\Omega}\times\vec{v}_{rel}+\vec{\Omega}\times(\vec{\Omega}\times\vec{R})\\
	\vec{a}_{rel}=\frac{\sum \vec{F}}{m}+\frac{-2m\vec{\Omega}\times\vec{v}_{rel}-m\vec{\Omega}\times(\vec{\Omega}\times\vec{R})}{m}\label{arel}.
\end{align}
Pertanto, è possibile considerare $\vec{a}_{rel}$, ovvero quella del sistema non inerziale solidale alla Terra, a patto di aggiungere il secondo termine alle forze:
\begin{equation}
	\vec{a}=\frac{\sum\vec{F}_{rel}+\sum\vec{F}_{app}}{m}.
\end{equation}
\subsection{Forza centrifuga}
Il termine $\vec{\Omega}\times(\vec{\Omega}\times\vec{R})$ rappresenta la forza centrifuga. Quest'ultima vale in modulo  $|F_c|=\Omega^2 R\sin(\frac{\pi}{2}-\theta)$ ed è diretta ortogonalmente all'asse di rotazione terrestre, in verso uscente.

L'espressione della forza centrifuga, proiettata sugli assi, è
\begin{equation}\label{scomp_fc}
	-\vec{\Omega}\times(\vec{\Omega}\times\vec{R})=-\vec{\Omega}^2\vec{R}\sin\theta\cos\theta\hat{u}_\theta+\cos^2\theta\hat{u}_z.
\end{equation}
Si noti che il segno negativo a sinistra dell'uguale è coerente con la \eqref{arel}.

Dalla \eqref{scomp_fc} si evince che il secondo termine, massimo all'equatore, ha l'effetto di diminuire la forza gravitazionale; è da tener presente che l'ordine di grandezza è di $10^{-3}$ m/s$^2$, mentre le forze di pressione verticale sono dell'ordine di 10 m/s$^2$: pertanto, è del tutto trascurabile.

Sulla direzione y ($\hat{u}_\theta$ è diretto lungo tale asse), invece, non sono presenti forze così intense e l'effetto di $F_c$ è di spostare le masse verso l'equatore (il primo termine nella \eqref{scomp_fc}). La forza centrifuga esiste da miliardi di anni e la sua continua azione è risultata in una progressiva modificazione della forma del pianeta: la superficie terrestre si è allineata ad una  equipotenziale per la forza gravitazionale e centrifuga. In questo modo è stata contrastata la tendenza a spostare le masse verso l'equatore (DETTA BENE? CORRETTO?).

\subsection{Coriolis}
La forza di Coriolis è\footnote{Espressa con i segni coerenti con la \eqref{arel}} data da
\begin{align}
	-2\vec{\Omega}\times\vec{v}&=-2[(\vec{\Omega}_{\parallel}+\vec{\Omega}_\perp)\times(\vec{v}_\parallel+\vec{v}_\perp)]=\\
	&=-2[(\vec{\Omega}_\parallel\times\vec{v}_\parallel)+(\vec{\Omega}_\perp\times\vec{\Omega}_\parallel)+(\vec{\Omega}_\parallel\times\vec{\Omega}_\perp)+(\vec{\Omega}_\perp\times\vec{\Omega}_\perp)]
\end{align}
dove il primo e il quarto termine sono nulli per le proprietà del prodotto vettoriale. Inoltre, essendo $||\vec{v}||\approx||\vec{v}_\parallel||$ (ovvero $||\vec{v}_\parallel||>>||\vec{v}_\perp||$), si ha:
\begin{equation}\label{coriolis}
	\frac{(\vec{F_{cor}})_{horiz}}{m}\approx-2\vec{\Omega}_\perp\times\vec{v}_\parallel.
\end{equation}

Il modulo della \eqref{coriolis} è $2\Omega\sin\theta$; la direzione  è perpendicolare a $\vec{v}_\parallel$ e nel piano tangente alla Terra; il verso è a destra del vettore velocità. È da notare che il modulo è dello stesso ordine di grandezza di quello della forza di pressione: la differenza tra le due, invece, funge da forza centripeta.

\section{Forze di pressione, cambio di variabili}
Nell'equazione per le forze di pressione \eqref{fpress}, la variabile dipendente risulta la pressione, mentre quelle indipendenti sono le coordinate spaziali. Restringendosi alle componenti orizzontali\footnote{Si è visto come la componente verticale è organizzata in modo da compensare la forza gravitazionale: per lo studio della dinamica dell'atmosfera, è di particolare interesse la proiezione orizzontale.}, infatti, si ha:
\begin{equation}\label{forizz}
	\frac{(\vec{F}_P)_{horiz}}{m}=-\frac{1}{\rho}\left(\frac{\partial P}{\partial x},\frac{\partial P}{\partial y}\right).
\end{equation}

Sperimentalmente, nei rilevamenti delle termosonde, la situazione è invertita: la variabile indipendente dei sondaggi è la pressione da cui viene ricavata la quota. È necessario effettuare un cambio di variabili in modo da poter calcolare le forze di pressione orizzontali dai dati dei termosondaggi.

Denotiamo la nuova variabile come ``$s$''. Si consideri una situazione come in figura (FARE), ove il punto A e B sono ad $x$ diverse e $z$ uguale, mentre il punto C si trova alla stessa $x$ di B ma a quota superiore. Allora, valutando le derivate con la nuova variabile $s$ mantenuta costante, si ha:
\begin{align}
	\left(\frac{\partial P}{\partial x}\right)_s&=\lim_{c\to a}\frac{\partial P}{\partial x}=\lim_{c\to a}\frac{P_c-P_a}{x_c-x_a}=\lim_{c\to a}\frac{P_c-P_b}{x_c-x_a}+\frac{P_b-P_a}{x_c-x_a}=\\
		&=\lim_{c\to a}\frac{(P_c-P_b)(z_c-z_b)}{(z_c-z_b)(x_c-x_a)}+\frac{P_b-P_a}{x_c-x_a}=\left(\frac{\partial P}{\partial x}\right)_x \left(\frac{\partial z}{\partial x}\right)_s+\left(\frac{\partial P}{\partial x}\right)_z,
\end{align}
dove si è tenuto conto che $z_b=z_a$. Inserendo $s=P=cost$ e utilizzando l'equazione dell'equilibrio idrostatico \eqref{idro} per la derivata rispetto a $z$, si ottiene:
\begin{equation}
	0=-\rho g\left(\frac{\partial x}{\partial x}\right)_P+\left(\frac{\partial P}{\partial x}\right)_z \Rightarrow \left(\frac{\partial P}{\partial x}\right)_z=\rho g \left(\frac{\partial z}{\partial x}\right)_P
\end{equation}

Dopo aver ripetuto il ragionamento per la coordinata $y$, è possibile esprimere l'equazione \eqref{forizz} nelle nuove coordinate $(x, y, P)$:
\begin{equation}\label{newcoord}
	\left(\frac{\vec{F}_P}{m}\right)_{horiz}=-g\left(\frac{\partial z}{\partial x}, \frac{\partial z}{\partial y}\right)_{P=cost}
\end{equation}

\section{Gradienti di temperatura al suolo}
Si ipotizzi una situazione come quella descritta in figura (FARE): due zone limitrofe a quota zero si trovano a temperature diverse. La pressione al suolo è uguale ma, come si evince dall'equazione \eqref{pz}, l'andamento in quota diverge a causa della diversa temperatura.

Ad esempio, si consideri un'estesa area geografica con pressione al suolo costante, isoterme parallele all'asse $x$ e un importante gradiente di temperatura lungo la $y$: 
\begin{equation}
	\frac{\partial T}{\partial y}=a.
\end{equation}
L'andamento della temperatura dipende, dunque, dalle coordinate $y$ e $z$:
\begin{equation}
	T(y,z)=T_0+ay-\gamma z
\end{equation}
con $\gamma=6.5\degree$C/km. Nell'ipotesi che la temperatura si riduca linearmente con $z$, è possibile ricavare l'andamento della pressione; nonché invertire la relazione per esplicitare $z=z(x,y,P)$ in modo da calcolare le forze di pressione con la \eqref{newcoord}:
\begin{align}
&\frac{P}{P_0}=\left(\frac{T_0+ay-\gamma z}{T_0+ay}\right)^{\frac{gM}{R\gamma}}\\
&\left(\frac{P}{P_0}\right)^{\frac{R\gamma}{gM}}=1-\frac{\gamma z}{T_0+ay}\\
&z=\frac{T_0+ay}{\gamma}\left[1-\left(\frac{P}{P_0}\right)^{\frac{R\gamma}{Mg}}\right].
\end{align}
La variazione di $z$ rispetto alle due coordinate orizzontali è
\begin{equation}
	\frac{\partial z}{\partial x}=0 \qquad \frac{\partial z}{\partial y}=\frac{a}{\gamma}\left[1-\left(\frac{P}{P_0}\right)^{\frac{R\gamma}{Mg}}\right]
\end{equation}
ovvero, utilizzando la \eqref{newcoord}, la forza di pressione diretta lungo l'asse $y$ è:
\begin{equation}
	(\vec{F}_P)_y=-\frac{ag}{\gamma}\left[1-\left(\frac{P}{P_0}\right)^{\frac{R\gamma}{Mg}}\right].
\end{equation}

Ad un gradiente di temperatura al suolo, l'atmosfera risponde con un gradiente di pressione che cresce all'aumentare della quota. Quest'ultimo causa intese forze di pressione che, per essere bilanciate dalla forza di Coriolis, sviluppano venti estremamente intensi. 

Ad esempio, si consideri una pressione al suolo $P_0=1000$ hPa e un gradiente di temperatura di
\begin{equation}
\frac{\partial T}{\partial y}=-\frac{10\degree\text{C}}{500\text{km}},
\end{equation}
si ricavano i seguenti valori:
\begin{table}[h]
	\centering
	\begin{tabular}{c|c|c}
		$P$(hPa) & $(\vec{F}_P)_y$ (m/s$^2$) & $v$ (m/s)  \\ \hline
		1000 & 0 & 0 \\ \hline
		850  & $0.9 \cdot 10^{-3}$ & 9 \\ \hline
		700  & $2 \cdot 10^{-3}$ & 20 \\ \hline
		500  & $4\cdot 10^{-3}$ &40 \\ \hline
		200  & $8\cdot 10^{-3}$ & 80 
	
	\end{tabular}
\end{table}

\section{Vorticità}
Dato un campo vettoriale $\vec{v}(x,y,z)=(u, v, w)$, si definisce \emph{vorticità} il rotore del campo:
\begin{equation}
	\nabla \times \vec{v}=\hat{u}_\lambda\left(\frac{\partial w}{\partial y}- \frac{\partial v}{\partial z}\right) 
	+\hat{u}_\theta\left(-\frac{\partial w}{\partial x}+\frac{\partial u}{\partial z}\right)
	+\hat{u}_z\left(\frac{\partial v}{\partial x}-\frac{\partial u}{\partial y}\right).
\end{equation}

Ognuno dei termini a destra dell'uguale indica quanto rapidamente il campo ruota in senso antiorario attorno ai versori $\hat{u}$.

Per l'atmosfera, è difficile ruotare nei piani $(x,z)$ e $(y,z)$ in quanto, in molte situazioni, il gradiente termico verticale ne impedisce lo spostamento di grandi masse d'aria. Pertanto, i termini $\hat{u}_\lambda$ e $\hat{u}_\theta$ del gradiente sono spesso poco importanti e trascurabili, mentre domina $\mathcal{Z}:=(\nabla\times\vec{v})_z$. Approssimando una colonna d'aria ad un corpo rigido che ruota attorno all'asse $z$, si ha:
\begin{equation}
	\mathcal{Z}=\frac{Rw-(-Rw)}{2R}-\frac{(-Rw)-Rw}{2R}=2w.
\end{equation}
Se la colonna, idealizzata come un cilindro, è costretta a cambiare altezza (per esempio incontrando una catena montuosa da scavalcare), dal momento che il momento angolare si deve conservare, segue che
\begin{equation}
	L_z=Iw=\frac{1}{2}mR^2_1w_1=\frac{1}{2}mR^2_2w_2,
\end{equation}
e dividendo per il volume di entrambi i cilindri:
\begin{equation}
	\frac{w_1}{h_1}=\frac{w_2}{h_2}.
\end{equation}
Pertanto, una circolazione ciclonica può essere indotta dal fatto che la massa d'aria si allunga verticalmente. 

Inoltre, è necessario tenere in considerazione che la Terra è un sistema di riferimento non inerziale: oltre alla vorticità della massa d'aria bisogna tenere conto della rotazione terrestre nelle quantità conservate. Detta $\mathcal{F}_0=2\Omega\sin\theta$ la vorticità della Terra, la quantità conservata è:
\begin{equation}
	\frac{\mathcal{Z}+\mathcal{F}_0}{h}=\text{const}.
\end{equation}

La vorticità della Terra è nulla all'equatore e massima ai poli: può apparire che masse d'aria si mettano in rotazione da sole, ma il fenomeno è dovuto al cambiamento di $h$ o di $\mathcal{F}_0$.

Ad esempio, si supponga $h$ costante e si ipotizzi che una massa d'aria sia costretta a muoversi verso sud: $\mathcal{F}_0$ diminuisce e, per conservare la vorticità totale, $\mathcal{Z}$ aumenta e la massa d'aria si mette in rotazione rispetto alla terra con una circolazione ciclonica. L'esatto opposto avviene se è costretta a muoversi a nord. Tale fenomeno è osservabile nel flusso zonale: questo è un flusso da ovest ad est che avviene lungo i paralleli. In esso si intuiscono oscillazioni attorno ai paralleli dovute alla conservazione della vorticità (INSERIRE FIGURA).